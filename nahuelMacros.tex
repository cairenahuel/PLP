\usepackage{changepage}
\usepackage{xcolor}
\usepackage[T1]{fontenc}
\usepackage{listings}
\usepackage{color}
\usepackage[top=1cm,bottom=2cm,left=1cm,right=1cm]{geometry}%

\definecolor{dkgreen}{rgb}{0,0.6,0}
\definecolor{gray}{rgb}{0.5,0.5,0.5}
\definecolor{mauve}{rgb}{0.58,0,0.82}

\lstset{frame=tb,
  language=Java,
  aboveskip=3mm,
  belowskip=3mm,
  showstringspaces=false,
  columns=flexible,
  basicstyle={\small\ttfamily},
  numbers=none,
  numberstyle=\tiny\color{gray},
  keywordstyle=\color{blue},
  commentstyle=\color{dkgreen},
  stringstyle=\color{mauve},
  breaklines=true,
  breakatwhitespace=true,
  tabsize=3
}
\renewcommand*\ttdefault{txtt}
\renewcommand*\familydefault{\ttdefault} %% Only if the base font of the document is to be typewriter style
\newcommand{\tbft}[2]{\par\addvspace{\baselineskip}\textbf{#1}\hspace{0.35em}{#2}\\\par\addvspace{\baselineskip}}
\newcommand{\ejercicio}[2]{\par\addvspace{\baselineskip}\textbf{Ejercicio #1.}\hspace{0.35em}{#2}\\\par\addvspace{\baselineskip}}
% \ejercicio{NUMERO}{ENUNCIADO}
%   Devuelve
% Ejercicio NUMERO. ---- ENUNCIADO ----
%
%formato
\newcommand{\salto}[1]{\par\addvspace{#1}}
\newcommand{\rojo}[1]{{\color{red}#1}}
\newcommand{\anotacion}[2][red]{\salto{1ex}\noindent\texttt{\color{#1}#2}\salto{1ex}}
\newcommand{\anotacionns}[2][red]{\noindent\texttt{\color{#1}#2}}
%
% si y solo si corto y largo
\newcommand{\sii}{\leftrightarrow}
\newcommand{\siiLargo}{\longleftrightarrow}
\newcommand{\slr}[1]{\ensuremath{\langle #1\rangle}}
\newcommand{\smm}[1]{\textless #1\textgreater}
\newcommand{\encabezadoTAD}[1]{\par\salto{1ex}\noindent TAD\ \ \normalfont\ttfamily#1 }
\newenvironment{tad}[1]{
\newcommand{\nombretad}{{\ttfamily#1}}
\newcommand{\nt}{\nombretad}
\newcommand{\obs}[2]{\par\noindent{\ttfamily obs} ##1: ##2\par}
\encabezadoTAD{#1}\{
    \begin{adjustwidth}{3em}{0em}}
{\end{adjustwidth}\par\}}
% \begin{tad}{nombre del tad}
%   AGREGAR UN OBSERVADOR
%     \obs{nombre observador}{tipo}
%     \nombretad <---------- DEVUELVE EL NOMBRE DEL TAD (du)
%
%   y aca se pueden usar todos los procs y cosas de la catedra
%
%     \end{tad}
\newcommand{\encabezadoImpl}[3]{\par\salto{1ex}\noindent \textbf{impl}\ \normalfont\ttfamily#1(#2):#3}
\newcommand{\asg}[2]{\salto{0em}\noindent{\ttfamily #1}:= #2\salto{0em}}
\newcommand{\asgns}[2]{\noindent{\ttfamily #1}:= #2}
\newenvironment{impl}[3]{\encabezadoImpl{#1}{#2}{#3}\{
\begin{adjustwidth}{3em}{0em}
}
{\end{adjustwidth}\par\}}
\newcommand{\encabezadoDesign}[2]{\par\salto{1ex}\noindent \textbf{modulo}\ {\normalfont\ttfamily #1} \textbf{implementa}{ \normalfont\ttfamily #2}}
\newenvironment{design}[2]{\encabezadoDesign{#1}{#2}\{
\newcommand{\var}[2]{\salto{0em}\noindent{\normalfont \bfseries var} \texttt{##1}: \texttt{##2}\salto{0em}}
\begin{adjustwidth}{3em}{0em}
}
{\end{adjustwidth}\}}
\newcommand{\ifthel}[3]{\salto{0ex}
\noindent{\normalfont \bfseries if}{ #1 }{\normalfont \bfseries then}\salto{0ex}
\noindent{\begin{adjustwidth}{1.5em}{0em}#2\end{adjustwidth}}\salto{0ex}
\noindent{\normalfont \bfseries else}\begin{adjustwidth}{1.5em}{0em}{#3}\end{adjustwidth}\salto{0ex}
\noindent{\bfseries end if}}\salto{0ex}
\newcommand{\while}[2]{
    \salto{0em}\noindent
    {\normalfont \bfseries while} \ensuremath{#1} \textbf{do}
    \begin{adjustwidth}{1.5em}{0em}{#2}\end{adjustwidth}\salto{0em}\noindent{\normalfont \bfseries end while}
}
\newcommand{\invrep}[2]{\salto{0em}\noindent{\bfseries pred} InvRep (#1)\{\salto{0em}\noindent\begin{adjustwidth}{2em}{0em}{#2}\end{adjustwidth}\salto{0em}\noindent\}}
\newcommand{\abs}[2]{\salto{0em}\noindent{\bfseries pred} Abs (#1)\{\salto{0em}\noindent\begin{adjustwidth}{2em}{0em}{#2}\end{adjustwidth}\salto{0em}\noindent\}}
